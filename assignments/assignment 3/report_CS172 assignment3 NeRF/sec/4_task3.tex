\section{Task3: Improve the performance by reasonably combining NGP and TensoRF}

We can merge the accelaration methods such as TensoRF and the Instant-NGP, take the advantages of them in order to chase a better performance. Such as a faster training time, or lower memory consuption, or better behavior(the metrics PSNR, SSIM, LPIPS, etc.).

% --------------------------------
\subsection{implement method}

Since TensoRF is implementing VM decomposing algorithm to decompose tensors, and Instant-NGP with hash encoding also generate a feature, with could be seen as a tensor. So with such inspiration, we can try to add method to decompose feature vector that were generated by Instant-NGP with VM mentioned in TensoRF.

Due to the time limit, the method were implemented with a naive implement, it may include much bugs or parameters to adjustment to achieve better performance.

% --------------------------------
\subsection{basic settings and way to run code}
Since the improved method was implemented on the base of Instant-NGP, so all settings were same with the previous work \ref{basic settings}. And to run the code, it is in the same jupyter notebook mentioned in \ref{way to run code}. Just run the module of the \textit{improved method} would work.

% --------------------------------
\subsection{results}

The result images were also shown in figure \ref{result}. And the training time, memory usage, metrics(PSNR, SSIM, LPIPS) were also shown in the table of section \ref{conclusion}.